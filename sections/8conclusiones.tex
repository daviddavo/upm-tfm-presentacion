\section{Conclusiones}

\begin{frame}{Trabajo realizado}
    \begin{itemize}
        \item Dataset
        \item Línea base: \textit{OpenPop}
        \item Evaluación: división en entrenamiento y prueba
        \item Desarrollo de modelos
        \item Elección de hiperparámetros
    \end{itemize}
\end{frame}

\begin{frame}{Limitaciones}
    \begin{itemize}
        \item Evaluación offline \textrightarrow ¿Cómo se comportará en realidad?
        \item Modelo GNN re-entrenado cada fold
        \item Simulación discretizada
    \end{itemize} 
\end{frame}

\begin{frame}{Trabajo futuro}
    \begin{itemize}
        \item Exploración de otros modelos
        \item Tener en cuenta el tiempo
        \item Expandir el grafo utilizado
        \item Evitar re-entrenamiento
        \item Mejorar la evaluación
        \item Probar otras maneras de elegir hiperparámetros
    \end{itemize}
\end{frame}

\begingroup
\setbeamertemplate{logo}{}
\begin{frame}{Preguntas}
    \vfill
    \begin{itemize}
        \item Código: \faGithub\ \href{https://github.com/daviddavo/upm-tfm-notebooks}{daviddavo/upm-tfm-notebooks}
        \item Diapositivas: \faGithub\ \href{https://github.com/daviddavo/upm-tfm-presentacion}{daviddavo/upm-tfm-presentacion}
    \end{itemize}
    \vfill
    \vspace{25mm}
    \doclicenseThis
\end{frame}
\endgroup
